\documentclass[twocolumn]{article}
\usepackage{tikz}
\usepackage{amsmath}
\usepackage[landscape,margin=2cm]{geometry}
\usepackage{siunitx}
\usepackage{listings}
\setlength{\columnsep}{2cm}
\begin{document}

\begin{lstlisting}
Java Learning Experience
Things that still need to be learned:
packages
generics
standardization (ISO?)
commonly used exceptions and errors
annotations

Nonfinal methods of Object (common to all classes)
equals (logical equality): reflexive, symmetric, transitive, consistent, x.equals(null) == false
hashCode
contract: x.equals(y) => x.hashcode() == y.hashcode()
The reverse is not necessarily true.
Good hash functions produce unequal hash codes for unequal objects.
toString: diagnostic message containing all the useful information in the class
clone
finalize (do not ever use)

Java Singleton Recipe:
public enum Susan {
	INSTANCE;
	private String nickname = "Susan";
	void setNickname(String nn) {if (nn == null) throw new NullReferenceError; nickname = nn;}
	String talk() {System.out.println(nickname + " the singleton");}
}
...
Susan.INSTANCE.setNickname("Sussy");
Susan.INSTANCE.talk(); // How to finalize a singleton?

Java Static Class Recipe:
public class UtilityClass {
	// kill the constructor
	private UtilityClass() {throw new AssertionError();}
	// static methods
	public static int randomNumber() {return 4;}
}

Java equals override recipe:
@Override public boolean equals(Object o) {
	if (this == o) // performance optimization
		return true;
	if (!(o instanceOf MyType)) // type check + null check
		return false;
	MyType mt = o; // cast
	return field.equals(o.field); // recursion, may require null check, special treatment for Float

Questions
How do you finalize a singleton? Consider a small program with a Logger singleton. A string is written to the log via Logger.INSTANCE.write("some log message"). The logger holds a file handle. How should the singleton close the log file at the end of the program? Two possible ways I can think of are: override finalize method and expose cleanup method. The book is very adment on never using the finalize method.
About lack of const. I had the same issue with C#. When a function takes an object as a parameter, there is no guarantee the function won't mutate the object. Coming from a C++ (with the const qualifier) background, this makes me nervous. Usually it's clear from the function's name what it does, but is there any other convention?
Is there any nice way to generate a hash?

JUnit Assertions (generally the first parameter is an optional message), (comparisons use (expected, actual))
assertArrayEquals
assertEquals (uses equals method)
assert[Not]Same (compares references)
assert[Not]Null
assertFalse / assertTrue
fail
\end{lstlisting}

\end{document}
