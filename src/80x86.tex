\documentclass[8pt]{article}
\usepackage{tikz}
\usepackage{amsmath}
\usepackage[landscape,margin=1cm]{geometry}
\usepackage{siunitx}
\usepackage{listings}
\usepackage{hyperref}
\usepackage{multicol}
\setlength{\columnsep}{1cm}

\pagenumbering{gobble}
\begin{document}
\begin{multicols}{5}

\section*{80x86 Assembly (NASM Syntax)}

Note: NASM sytax is most similar to Intel Syntax.

\begin{lstlisting}
# Elf binary
nasm -f elf test.s
ld -melf_i386 test.o
# Raw assembly
nasm test.s
# Debug symbols
nasm -g -F dward ...
# Optimize offsets
nasm -O1 ...
# TODO: disassemble
\end{lstlisting}

\subsection*{Registers}

\begin{lstlisting}
eax-edx: general purpose 32-bit
ax-dx: general purpose 16-bit (lower half of e[abcd]x)
al-dl: general purpose 8-bit (lower half of [abcd]x)
ah-dh: general purpose 8-bit (higher half of [abcd]x)
esp: stack pointer (points to top element of stack)
ebp: base pointer / frame pointer
esi / edi: source and destination index (we did not need these for our compiler)
eip: instruction pointer
\end{lstlisting}

% TODO: diagram of registers
% TODO: diagram of stack layout

\subsection*{Instructions}

\begin{lstlisting}
nop
ret
cdq
idiv X
jmp X
call X
push X
pop X
int 0x80
dd X
dw X
db X
je
jne
jg
jge
jl
jle
ja
jae
jb
jbe
\end{lstlisting}

Use ";" to start a line comment.

\subsection*{Assembler Directives}

\begin{lstlisting}
db
dw
dd

resb

'C:\\files'
`string\n`

section .bss
section .text
section .rodata
section .data
\end{lstlisting}

\subsection*{Calling Convention}

\subsection*{Linux Syscall Convention}


Minimal usermode:

\begin{lstlisting}
global _start
_start:
 ; write(stdout, &str, 7)
 mov eax, 4
 mov ebx, 1
 mov ecx, str
 mov edx, 7
 int 0x80
 ; exit(0)
 mov eax, 1
 mov ebx, 0
 int 0x80
section .rodata
str: db `Hello!\n`
\end{lstlisting}

\subsection*{Serial Port}

See: \url{https://wiki.osdev.org/Serial_Ports}

Minimal kernelmode:

\begin{lstlisting}
global _start
_start:
 mov dx, 0x3f8+5 ; status
 waitRx:
  in al, dx
  test al, 0x01
  je waitRx
 mov dx, 0x3f8 ; data
 in al, dx
 ; Capitalize
  cmp al, 'a'
  jl skip
  cmp al, 'z'
  jg skip
  or al, 0x20
  skip:
 waitTx:
  mov dx, 0x3f8+5 ; status
  in al, dx
  test al, 0x20
  je waitTx
 mov dx, 0x3f8 ; data
 out dx, al
 jmp _start
\end{lstlisting}

\end{multicols}
\end{document}
